\documentclass[margin,line]{res}

\usepackage{amsmath,latexsym}
\usepackage{multicol}

\oddsidemargin -0.5in
\evensidemargin -0.5in %-.5in
\topmargin -0.02 in
\textwidth=6in
\addtolength{\textheight}{0.5in}

\itemsep=0in
\parsep=0in

\newenvironment{list1}{
  \begin{list}{\ding{113}}{%
      \setlength{\itemsep}{0in}
      \setlength{\parsep}{0in} \setlength{\parskip}{0in}
      \setlength{\topsep}{0in} \setlength{\partopsep}{0in}
      \setlength{\leftmargin}{0.17in}}}{\end{list}}
\newenvironment{list2}{
  \begin{list}{$\ddagger$}{%
      \setlength{\itemsep}{0in}
      \setlength{\parsep}{0in} \setlength{\parskip}{0in}
      \setlength{\topsep}{0in} \setlength{\partopsep}{0in}
      \setlength{\leftmargin}{0.2in}}}{\end{list}}

%% Usage
% Names, ``Title'', Journal, Month Year, doi.
\newcommand{\article}[5]{%
#1, ``#2'', {\it #3}, #4, #5.%
}

\newcommand{\submitted}[4]{%
#1, ``#2'', {\it Submitted to #3}, #4.%
}
%\newcommand{\submitted}[3]{%
%#1,``#2'',{\it Submitted},#3.%
%}

%% Usage
% Names, ``Title'',Conference, City, Publisher, {Month Day, Year}, pages.
\newcommand{\inproceeding}[7]{%
#1, ``#2'', {\it #3}, #4, #5, #6, #7.%
}

%% Usage
% Names, ``Title'',Conference, {Month Day, Year}.
\newcommand{\talk}[4]{%
#1,``#2'', {\it #3}, #4.%
}

\begin{document} 
\name{Maximilian H. Bremer \vspace*{.1in}}

\begin{resume}

\section{\sc Contact Information} 
%vspace{.05in}
\begin{tabular}{@{}p{3.25in}p{4in}}
201 East 24th St, Stop C0200 & { Cell:} (214) 862-7370 \\
Austin, TX 78712-1229 USA & {E-mail:} {\tt max@ices.utexas.edu}
\end{tabular} 


%%%%%%%%%%%%%%%%%%%%%%%%%%%%%%%%%%
\section{\sc Education} 
%%%%%%%%%%%%%%%%%%%%%%%%%%%%%%%%%

\textbf{University of Texas at Austin}, \\
\textbf{Institute for Computational Engineering and Sciences} \\ 
\vspace*{-.1in} 

\begin{list1} 
\item[] Ph.D. Candidate, Computational Science, Engineering, and Mathematics.
\item[] M.Sc., Computational Science, Engineering, and Mathematics, May 2018.
\item[] Advisor: Clint Dawson
\item[] Expected Graduation Date: Summer 2020
\end{list1} 

\textbf{University of Cambridge}, \\
\vspace*{-.1in} 

\begin{list1} 
\item[] M.A.St., Part III Pure Mathematics, June 2015
\item[] Emphasis: Partial Differential Equations / Analysis
\end{list1} 

\textbf{The University of Texas at Austin} \\

\vspace*{-.1in}
\begin{list1}
\item[] B.Sc., Aerospace Engineering, May 2014.
\item[] B.Sc., Applied Mathematics, May 2014. 
%\item[] Senior Thesis: \textit{A Discontinuous Galerkin Local Timestepping Scheme for the Prediction of Hurricane Storm Surge}
\end{list1} 


%%%%%%%%%%%%%%%%%%%%%%%%%%%%%%%%%%
\section{\sc Research Experience}
%%%%%%%%%%%%%%%%%%%%%%%%%%%%%%%%%%
\textit{Research Interests}
\vspace{0.05in}
\begin{list2}
\item Developing an asynchronous HPX-based implementation of a finite element model for hurricane storm surge simulation, 	(Principal Investigator: Clint Dawson).
\item Using reinforcement learning to derive load balancing strategies for hurricane storm surge modeling, 	(Principal Investigator: Clint Dawson).

%\item Developing load balancing strategies for hurricane storm surge simulations with asynchronous run-times,
%	(Principal Investigators: Cy Chan, John Bachan).
\end{list2}

%\textbf{Lawrence Berkeley National Lab} \hfill June 2016--August 2016\\
%\textit{Computer Architecture Group} 
%\vspace{0.05in}
%\begin{list2}
%\item[] Examined load balancing strategies for asynchronous execution models of hurricane storm surge simulations. Developed and validated a simulator of the application code upon which specific strategies were then examined.
%\end{list2} 

%\textbf{German Aerospace Center} \hfill June 2014--September 2014\\
%\textit{Center for Computer Applications in Aerospace Science and Engineering}
%\vspace{0.05in}
%\begin{list2}
%\item[] Analyzed the effects of element-type on discontinuous Galerkin solutions for aerospace applications. Implemented I/O features for finite element code using multigrid to solve steady Navier-Stokes with some turbulence model (e.g. $k- \omega$ or SA-neg). Validated data shown on NASA Langley's Turbulence Modeling Resource.
%\end{list2}

%\textbf{ Max Planck Institute} \hfill May 2012--July 2012\\
%\textit{Computational Methods in Systems and Controls Theory Group}
%\vspace{0.05in}
%\begin{list2}
%\item[] Developed algorithms for computation of generalized pseudospectra, exhibiting acceleration factors of over 200 compared to na\"ive implementations. Applications included analyzing robustness of the Brazilian power network using $\mathcal{H}_{\infty}$ norms.
%\end{list2} 

%\textbf{The University of Texas at Austin} \hfill June 2013--July 2015\\
%\textit{Computational Hydraulics Group}
%\vspace{0.05in}
%\begin{list2}
%\item[] Used Runge-Kutta Discontinuous Galerkin methods to simulate the effects of rainfall-induced flooding through storm sewer systems using the shallow water equations and kinematic wave equation to improve the accuracy of hurricane storm surge predictions. Included independent study of LeVeque's \textit{Numerical Methods for Conservation Laws}.
%\end{list2}

%%%%%%%%%%%%%%%%%%%%%%%%%%%%%
\section{\sc Publications}
%%%%%%%%%%%%%%%%%%%%%%%%%%%%%
\textit{Journal Articles}
\vspace{0.05in}
\begin{list2}
\item[1.] \submitted{{\bf M.B.}, Kazebek Kazhyken, Hartmut Kaiser, Craig Michoski, Clint Dawson}{Performance Comparison of HPX Versus Traditional Parallelization Strategies for the Discontinuous Galerkin Method}{J. Sci. Comput.}{2018}
\end{list2}
%\textit{Conference Papers}
%\vspace{0.05in}

%% Usage for \talk
% Names, ``Title'',Conference, {Month Day, Year}.
\textit{Presentations/Talks}
\vspace{0.05in}
\begin{list2}
\item[7.] \talk{{\bf M.B.}, Kazbek Kazhyken, Hartmut Kaiser, Craig Michoski, Clint Dawson}{Task-based Parallelism for Finite-Element Models of Shallow Water Flows}{World Congress in Computational Mechanics}{July 24, 2018}

\item[6.] \talk{{\bf M.B.}}{Computational Modeling of Hurricane Storm Surge}{Harrington Annual Research Symposium}{April 10, 2018}

\item[5.] \talk{{\bf M.B.}, Zach Byerly, Hartmut Kaiser, Craig Michoski, Clint Dawson}{Performance Comparison of HPX versus Traditional Parallelization Models for Finite-Element Models of Environmental Flows}{American Meteorological Society Annual Meeting}{January 10, 2018}

\item[4.] \talk{{\bf M.B.}}{Wrangling Concurrency with HPX}{ICES Seminar--Student Forum}{December 8, 2017}

\item[3.] \talk{{\bf M.B.}, Craig Michoski, Zach Byerly, Hartmut Kaiser, Clint Dawson}{Optimizing Discontinuous Galerkin Finite Element Kernels on Knights Landing Chips}{Texas Applied Mathematics and Engineering Symposium}{September 22, 2017}

\item[2.] \talk{{\bf M.B.}, John Bachan, Cy Chan}{Asynchronous Load Balancing for Hurricane Storm Surge Simulations}{LBL Computing Sciences Seminar}{February 16, 2017}

\item[1.] \talk{{\bf M.B.}, Clint Dawson, Zach Byerly, Hartmut Kaiser, Craig Michoski, Andreas Sch\"afer}{Application of High Performance ParallelX (HPX) for High Performance Computing of Hurricane Storm Surge}{American Meteorological Society Annual Meeting}{January 25, 2017}
\end{list2}
%%%%%%%%%%%%%%%%%%%%%%%%%%%%%
\section{\sc Honors and \\ Awards}
%%%%%%%%%%%%%%%%%%%%%%%%%%%%%

\textit{Honors and Awards}
\vspace{0.05in}
\begin{list2}
\item[] Department of Energy Computational Science Graduate Fellowship (2015)
\item[] Donald D. Harrington Fellowship (2015)
\item[] Cockrell School of Engineering Outstanding Scholar/Leader Award (2014)
\item[] Graham F. Carey Scholarship in Computational Science (2013)
\end{list2}


%%%%%%%%%%%%%%%%%%%%%%%%%%%%%%%%%
%\section{\sc Internships and Positions}
%%%%%%%%%%%%%%%%%%%%%%%%%%%%%%%%%%

%\textit{Professional and Honorary Societies}
%\vspace{0.05in}
%\begin{list2} 
%\item[] Society for Industrial and Applied Mathematics (SIAM)
%\end{list2} 


%%%%%%%%%%%%%%%%%%%%%%%%%%%%%%%%%
\section{\sc Computer \\ Skills}
%%%%%%%%%%%%%%%%%%%%%%%%%%%%%%%%%

\textit{Languages}
\vspace{0.05in}
\begin{list2}
\item C++, Python, FORTRAN, MatLab, \LaTeX
\item MPI, TensorFlow
\end{list2}
%\textit{Areas of Exposure}
%\vspace{0.05in}
%\begin{list2}
%\item Computer Networks, Data Mining, Systems \& Hardware, Data
% Structures.

% Computational Experimentation, Systems \& Hardware
%\end{list2}


%%%%%%%%%%%%%%%%%%%%%%%%%%%%%%%%%
%\section{\sc Other Skills}
%%%%%%%%%%%%%%%%%%%%%%%%%%%%%%%%

%\textit{Languages}
%\vspace{0.05in}
%\begin{list2}
%\item English, German
%\end{list2}
%\section{\sc *References upon} {\sc request}
\end{resume}
\end{document}
