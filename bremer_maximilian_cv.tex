\documentclass[margin,line]{res}

\usepackage{amsmath,latexsym}
\usepackage{multicol}
\usepackage{url}
\usepackage{hyperref}

\oddsidemargin -0.5in
\evensidemargin -0.5in %-.5in
\topmargin -0.02 in
\textwidth=6in
\addtolength{\textheight}{0.5in}

\itemsep=0in
\parsep=0in

\newenvironment{list1}{
  \begin{list}{\ding{113}}{%
      \setlength{\itemsep}{0in}
      \setlength{\parsep}{0in} \setlength{\parskip}{0in}
      \setlength{\topsep}{0in} \setlength{\partopsep}{0in}
      \setlength{\leftmargin}{0.17in}}}{\end{list}}
\newenvironment{list2}{
  \begin{list}{$\ddagger$}{%
      \setlength{\itemsep}{0in}
      \setlength{\parsep}{0in} \setlength{\parskip}{0in}
      \setlength{\topsep}{0in} \setlength{\partopsep}{0in}
      \setlength{\leftmargin}{0.2in}}}{\end{list}}

%% Usage
% Names, ``Title'', Journal, Month Year, doi.
\newcommand{\article}[5]{%
#1, ``#2'', {\it #3}, #4, #5.%
}

\newcommand{\submitted}[4]{%
#1, ``#2'', {\it Submitted to #3}, #4.%
}

\newcommand{\inpreparation}[3]{%
#1, ``#2'', {\it In preparation}, #3.
}

%% Usage
% Names, ``Title'',Conference, City, Publisher, {Month Day, Year}, pages.
\newcommand{\inproceeding}[7]{%
#1, ``#2'', {\it #3}, #4, #5, #6, #7 pages.%
}

%% Usage
% Names, ``Title'',Conference, {Month Day, Year}.
\newcommand{\talk}[4]{%
#1,``#2'', {\it #3}, #4.%
}

\def\keywordgap{\hspace{2pt minus.5pt}}
\newcommand\keywordsep{\keywordgap\textbullet\keywordgap}

\begin{document} 
\name{Maximilian H. Bremer \vspace*{.1in}}

\begin{resume}

\section{\sc Contact Information} 
%vspace{.05in}
\begin{tabular}{@{}p{3.25in}p{4in}}
201 East 24th St, Stop C0200 & { Cell:} (214) 862-7370 \\
Austin, TX 78712-1229 USA & {E-mail:} {\tt max@oden.utexas.edu}
\end{tabular} 


%%%%%%%%%%%%%%%%%%%%%%%%%%%%%%%%%%
\section{\sc Education} 
%%%%%%%%%%%%%%%%%%%%%%%%%%%%%%%%%

\textbf{University of Texas at Austin}, \\
\textbf{Oden Institute for Computational Engineering and Sciences} \\ 
\vspace*{-.1in} 

\begin{list1} 
\item[] Ph.D. Candidate, Computational Science, Engineering, and Mathematics.
\item[] M.Sc., Computational Science, Engineering, and Mathematics, May 2018.
\item[] Thesis: Task-Based Parallelism for Hurricane Storm Surge
\item[] Advisor: Clint Dawson
\item[] Expected Graduation Date: Summer 2020
\end{list1}

\textbf{University of Cambridge}, \\
\vspace*{-.1in}

\begin{list1}
\item[] M.A.St., Part III Pure Mathematics, June 2015.
\item[] Emphasis: Partial Differential Equations / Analysis
\end{list1}

\textbf{The University of Texas at Austin} \\

\vspace*{-.1in}
\begin{list1}
\item[] B.Sc., Aerospace Engineering, May 2014, GPA: 3.98/4.00.
\item[] B.Sc., Applied Mathematics, May 2014, GPA: 3.98/4.00.
%\item[] Senior Thesis: \textit{A Discontinuous Galerkin Local Timestepping Scheme for the Prediction of Hurricane Storm Surge}
\end{list1}


%%%%%%%%%%%%%%%%%%%%%%%%%%%%%%%%%%
\section{\sc Research Experience}
%%%%%%%%%%%%%%%%%%%%%%%%%%%%%%%%%%
\textit{Keywords}
\vspace{0.05in}
\begin{list1}
\item[] Task-based parallelism \keywordsep High performance computing \keywordsep Discontinuous Galerkin finite elements \keywordsep Shallow water equations \keywordsep Hurricane storm surge
\end{list1}
%\begin{list2}
%\item Developing an asynchronous HPX-based implementation of a finite element model for hurricane storm surge simulation, 	(Principal Investigator: Clint Dawson).
%\item Using reinforcement learning to derive load balancing strategies for hurricane storm surge modeling, 	(Principal Investigator: Clint Dawson).

%\item Developing load balancing strategies for hurricane storm surge simulations with asynchronous run-times,
%	(Principal Investigators: Cy Chan, John Bachan).
%\end{list2}

\textbf{The University of Texas at Austin} \hfill Fall 2015--present\\
\textit{Computational Hydraulics Group}
\hfill {\it Supervised by:} Clint Dawson\\
\vspace{-0.1in}
\begin{list2}
\item Compared and analyzed performance of flat (non-blocking) MPI versus an HPX-based parallelization on Knights Landing and Skylake architectures on Stampede2.
\item Performed roofline analyses and implemented vectorization strategies that accelerated the code by 2.9x.
\item Refactored in-house discontinuous Galerkin storm surge code to improve productivity. Introduced software engineering best practices, e.g. continuous integration and unit testing.
\end{list2}

\textbf{Lawrence Berkeley National Lab} \hfill Summer 2019\\
\textit{Computer Architecture Group} \hfill {\it Supervised by:} Cy Chan\\
\vspace{-0.1in}
\begin{list2}
\item Designed timestepping strategies that adaptively refine and coarsen timesteps to ensure that cells optimally satisfy the CFL condition.
\item Leveraged existing speculative parallel discrete event simulation techniques to efficiently parallelize the code.
\end{list2}

\textbf{Lawrence Berkeley National Lab} \hfill Summer 2016\\
\textit{Computer Architecture Group} \hfill {\it Supervised by:} Cy Chan\\
\vspace{-0.1in}
\begin{list2}
\item Explored load balancing strategies for asynchronously run hurricane storm surge simulations.
\item Developed and validated a discrete event simulator of the application code for rapid prototyping of load balancing strategies.
\end{list2}

%\textbf{German Aerospace Center} \hfill June 2014--September 2014\\
%\textit{Center for Computer Applications in Aerospace Science and Engineering}
%\vspace{0.05in}
%\begin{list2}
%\item[] Analyzed the effects of element-type on discontinuous Galerkin solutions for aerospace applications. Implemented I/O features for finite element code using multigrid to solve steady Navier-Stokes with some turbulence model (e.g. $k- \omega$ or SA-neg). Validated data shown on NASA Langley's Turbulence Modeling Resource.
%\end{list2}

%\textbf{ Max Planck Institute} \hfill May 2012--July 2012\\
%\textit{Computational Methods in Systems and Controls Theory Group}
%\vspace{0.05in}
%\begin{list2}
%\item[] Developed algorithms for computation of generalized pseudospectra, exhibiting acceleration factors of over 200 compared to na\"ive implementations. Applications included analyzing robustness of the Brazilian power network using $\mathcal{H}_{\infty}$ norms.
%\end{list2}

%\textbf{The University of Texas at Austin} \hfill June 2013--July 2015\\
%\textit{Computational Hydraulics Group}
%\vspace{0.05in}
%\begin{list2}
%\item[] Used Runge-Kutta Discontinuous Galerkin methods to simulate the effects of rainfall-induced flooding through storm sewer systems using the shallow water equations and kinematic wave equation to improve the accuracy of hurricane storm surge predictions. Included independent study of LeVeque's \textit{Numerical Methods for Conservation Laws}.
%\end{list2}

%%%%%%%%%%%%%%%%%%%%%%%%%%%%%
\section{\sc Publications}
%%%%%%%%%%%%%%%%%%%%%%%%%%%%%
\textit{Journal Articles}
\vspace{0.05in}
\begin{list2}
% Names, ``Title'', Journal, Month Year, doi.
\item[2.] \article{{\bf M.B.}, John Bachan, Cy Chan, Clint Dawson}{Adaptive Total Variation Stable Local Timestepping for Conservation Laws}{Submitted to SIAM Journal on Scientific Computing}{2020}{\newline \href{https://arxiv.org/abs/2003.09020}{arXiv:2003.09020 [math.NA]}}
\item[1.] \article{{\bf M.B.}, Kazbek Kazhyken, Hartmut Kaiser, Craig Michoski, Clint Dawson}{Performance Comparison of HPX Versus Traditional Parallelization Strategies for the Discontinuous Galerkin Method}{J. Sci. Comput.}{May 2019}{doi:10.1007/s10915-019-00960-z}
\end{list2}
%\clearpage
\textit{Conference Papers}
\vspace{0.05in}
\begin{list2}
\item[1.] \inproceeding{{\bf M.B.}, John Bachan, Cy Chan}{Semi-Static and Dynamic Load Balancing for Asynchronous Hurricane Storm Surge Simulations}{2018 IEEE/ACM Parallel Applications Workshop, Alternatives To MPI (PAW-ATM)}{Dallas, Texas}{IEEE}{November 16, 2018}{13}
\end{list2}

%% Usage for \talk
% Names, ``Title'',Conference, {Month Day, Year}.

\textit{Presentations/Talks}
\vspace{0.05in}
\begin{list2}

\item[13.] \talk{{\bf M.B.}, Cy Chan, John Bachan, Clint Dawson}{Adaptive Local Timestepping and its Parallelization}{SIAM Conference on Parallel Processing for Scientific Computing (PP20)}{February 15, 2020}

\item[12.] \talk{{\bf M.B.}, Cy Chan, John Bachan, Clint Dawson}{Adaptive Local Timestepping for Shallow Water Flows}{$18^{th}$ International Workshop on Multi-scale (Un)-structured Mesh Numerical Modeling for Coastal, Shelf, and Global Ocean Dynamics (IMUM)}{September 26, 2019}

\item[11.] \talk{Clint Dawson, {\bf M.B.}}{Vectorization of Discontinuous Galerkin Schemes for Shallow Water Flows}{U.S. National Congress on Computational Mechanics}{July 31, 2019}

\item[10.] \talk{{\bf M.B.}}{Simulation of Shallow Water Flows Using HPX}{DOE CSGF Program Review}{July 15, 2019}

\item[9.] \talk{{\bf M.B.}, Hartmut Kaiser, Clint Dawson}{Asynchronous Finite Element Simulation of Coastal Inundation}{SIAM Conference on Computational Science and Engineering (CSE19)}{February 28, 2019}

\item[8.] \talk{{\bf M.B.}, John Bachan, Cy Chan}{Semi-Static and Dynamic Load Balancing for Asynchronous Hurricane Storm Surge Simulations}{2018 IEEE/ACM Parallel Applications Workshop, Alternatives To MPI (PAW-ATM)}{November 16, 2018}

\item[7.] \talk{{\bf M.B.}, Kazbek Kazhyken, Hartmut Kaiser, Craig Michoski, Clint Dawson}{Task-based Parallelism for Finite-Element Models of Shallow Water Flows}{World Congress in Computational Mechanics}{July 24, 2018}

\item[6.] \talk{{\bf M.B.}}{Computational Modeling of Hurricane Storm Surge}{Harrington Annual Research Symposium}{April 10, 2018}

\item[5.] \talk{{\bf M.B.}, Zach Byerly, Hartmut Kaiser, Craig Michoski, Clint Dawson}{Performance Comparison of HPX versus Traditional Parallelization Models for Finite-Element Models of Environmental Flows}{American Meteorological Society Annual Meeting}{January 10, 2018}

\item[4.] \talk{{\bf M.B.}}{Wrangling Concurrency with HPX}{ICES Seminar--Student Forum}{December 8, 2017}

\item[3.] \talk{{\bf M.B.}, Craig Michoski, Zach Byerly, Hartmut Kaiser, Clint Dawson}{Optimizing Discontinuous Galerkin Finite Element Kernels on Knights Landing Chips}{Texas Applied Mathematics and Engineering Symposium}{September 22, 2017}

\item[2.] \talk{{\bf M.B.}, John Bachan, Cy Chan}{Asynchronous Load Balancing for Hurricane Storm Surge Simulations}{LBL Computing Sciences Seminar}{February 16, 2017}

\item[1.] \talk{{\bf M.B.}, Clint Dawson, Zach Byerly, Hartmut Kaiser, Craig Michoski, Andreas Sch\"afer}{Application of High Performance ParallelX (HPX) for High Performance Computing of Hurricane Storm Surge}{American Meteorological Society Annual Meeting}{January 25, 2017}
\end{list2}

%%%%%%%%%%%%%%%%%%%%%%%%%%%%%
\section{\sc Honors and \\ Awards}
%%%%%%%%%%%%%%%%%%%%%%%%%%%%%

\textit{Honors and Awards}
\vspace{0.05in}
\begin{list2}
\item[] Department of Energy Computational Science Graduate Fellowship (2015)
\item[] Donald D. Harrington Fellowship (2015)
\item[] Cockrell School of Engineering Outstanding Scholar/Leader Award (2014)
\item[] Graham F. Carey Scholarship in Computational Science (2013)
\end{list2}


%%%%%%%%%%%%%%%%%%%%%%%%%%%%%%%%%
%\section{\sc Internships and Positions}
%%%%%%%%%%%%%%%%%%%%%%%%%%%%%%%%%%

%\textit{Professional and Honorary Societies}
%\vspace{0.05in}
%\begin{list2} 
%\item[] Society for Industrial and Applied Mathematics (SIAM)
%\end{list2} 


%%%%%%%%%%%%%%%%%%%%%%%%%%%%%%%%%
\section{\sc Computer \\ Skills}
%%%%%%%%%%%%%%%%%%%%%%%%%%%%%%%%%

\textit{Languages}
\vspace{0.05in}
\begin{list2}
\item C++, Python, Bash Scripting, FORTRAN, MatLab, \LaTeX
\end{list2}

\textit{Software Development}\\
\vspace{0.05in}
{\bf DGSWEM-v2} \hfill \url{https://github.com/UT-CHG/dgswemv2}
\begin{list2}
\item Discontinuous Galerkin (DG) finite element code for the simulation of coastal flows.
\item Provides MPI+OpenMP and HPX parallelization back-ends.
\item License: MIT
\end{list2}

\textit{Areas of Exposure}
\vspace{0.05in}
\begin{list2}
\item {\em Packages}: MPI, OpenMP
\item {\em Libraries}: HPX, Eigen, Blaze, UPC++
\item {\em Software Engineering}: git, make, cmake, CircleCI, Docker
\end{list2}

%%%%%%%%%%%%%%%%%%%%%%%%%%%%%%%%%
\section{\sc Academic \\ Service}
%%%%%%%%%%%%%%%%%%%%%%%%%%%%%%%%

\textit{Conferences/Seminars Organized}
\vspace{0.05in}
\begin{list2}
\item Co-organizer, {\em ICES Seminar--Babu\v{s}ka Forum} \hfill Fall 2018--Spring 2019
\item Co-organizer, {\em Texas Applied Mathematics and Engineering Symposium} \hfill September 2017
\end{list2}

\textit{Societal Membership}
\vspace{0.05in}
\begin{list2}
\item CSEM Student Representative \hfill Fall 2018--Spring 2019
\item Society for Industrial and Applied Mathematics (SIAM)
\end{list2}

\textit{Reviewer, Journal Articles}
\vspace{0.05in}
\begin{list2}
\item Computer Methods in Applied Mechanics and Engineering
\item Journal of Computational Physics
\end{list2}
%\section{\sc *References upon} {\sc request}
\end{resume}
\vfill
\centerline{Last Updated: \today}
\end{document}
